% !TeX root = ../../a4.tex
% chktex-file 1

\subsection{Entrada}\label{subsection:hebdomada-sancta/dominica-in-palmis-de-passione-domini/introitus}
\MakeChantAntiphonPsalm{hosanna-filio-david.7a}{psalmi-ad-introitum}

\AllowPageBreak

\subsection{Procissão I}\label{subsection:hebdomada-sancta/dominica-in-palmis-de-passione-domini/ad-processionem-1}
\MakeChantAntiphonPsalm{pueri-hebraeorum-portantes-ramos.1f}{psalmi-ad-processionem}

\AllowPageFlush

\subsection{Procissão II}\label{subsection:hebdomada-sancta/dominica-in-palmis-de-passione-domini/ad-processionem-2}
\MakeChantAntiphonPsalm{pueri-hebraeorum-vestimenta.1f}{psalmi-ad-processionem}

\subsection{Hino a Cristo Rei}\label{subsection:hebdomada-sancta/dominica-in-palmis-de-passione-domini/hymnus-ad-christum-regem}
\MakeChantLongPsalm{hymni}{gloria-laus}{
  {psalmus-v1}{psalmus-v1-pt},
  {psalmus-v2}{psalmus-v2-pt},
  {psalmus-v3}{psalmus-v3-pt},
  {psalmus-v4}{psalmus-v4-pt},
  {psalmus-v5}{psalmus-v5-pt},
  {psalmus-v6}{psalmus-v6-pt}
}

\subsection{Chegada da procissão à igreja}\label{subsection:hebdomada-sancta/dominica-in-palmis-de-passione-domini/ingrediente-processione-in-ecclesiam}
\MakeChantAntiphonPsalm{hosanna-in-excelsis}{psalmi-ad-processionem}

\begin{rubrica}
  Seguem-se os cânticos para a missa do V Domingo do Tempo da Quaresma, página~\pageref{section:tempus-quadragesimae/dominica-5}, com exceção da antífona de entrada e seu salmo. Omite-se o Kyrie.
\end{rubrica}