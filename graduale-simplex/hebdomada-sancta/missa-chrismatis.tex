% !TeX root = ../../a4.tex
% chktex-file 1

\subsection{Entrada}\label{subsection:hebdomada-sancta/missa-chrismatis/introitus}
\MakeChantAntiphonPsalm{a-fructu-frumenti.1f}{psalmi-ad-introitum}

\subsection[Salmo Responsorial I]{Salmo Responsorial I \textmd{E 3}}\label{subsection:hebdomada-sancta/missa-chrismatis/psalmus-responsorius-1}
\MakeChantPsalmTwoVerses{psalmi-responsorii}{misericordias-domini.E3}

\AllowPageFlush

\subsection[Salmo Responsorial II]{Salmo Responsorial II \textmd{C 2 g}}\label{subsection:hebdomada-sancta/missa-chrismatis/psalmus-responsorius-2}
\MakeChantPsalmTwoVerses{psalmi-responsorii}{exsultate-iusti-in-domino.C2g}

\AllowPageBreak

\subsection{Ofertório}\label{subsection:hebdomada-sancta/missa-chrismatis/offertorium}
\MakeChantLongPsalm{hymni}{o-redemptor}{
  {psalmus-v1}{psalmus-v1-pt},
  {psalmus-v2}{psalmus-v2-pt},
  {psalmus-v3}{psalmus-v3-pt},
  {psalmus-v4}{psalmus-v4-pt},
  {psalmus-v5}{psalmus-v5-pt},
  {psalmus-v6}{psalmus-v6-pt},
  {psalmus-v7}{}
}

\AllowPageFlush

\subsection{Comunhão}\label{subsection:hebdomada-sancta/missa-chrismatis/communio}
\MakeChantAntiphonPsalm{dilexisti-iustitiam.8G}{psalmi-ad-communionem}