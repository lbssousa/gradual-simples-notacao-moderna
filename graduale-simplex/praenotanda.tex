% chktex-file 1
% chktex-file 19
\newcommand{\GR}{\emph{Graduale Romanum}}
\newcommand{\GS}{\emph{Graduale Simplex}}
\newcommand{\KS}{\emph{Kyriale Simplex}}
\newcommand{\Al}{\emph{Allelúia}}
\newcommand{\Schola}{\emph{schola}}
\newcommand{\ScholaC}{\emph{schola cantorum}}
\newcommand{\FirstPara}[1]{\noindent\textcolor{gregoriocolor}{#1.}}
\newcommand{\Para}[1]{\vspace{\baselineskip}\noindent\textcolor{gregoriocolor}{#1.}}

\tocchapter{Introdução}

\tocsection{Sobre este projeto}

\subsection{Motivação}

O {\GS} é o que temos hoje de mais acessível em matéria de canto gregoriano para uso no âmbito paroquial. A edição mais recente, publicada em 1975, atendendo a um pedido da Constituição \emph{Sacrosanctum Concilium}, é uma das mais fortes evidências de que o canto gregoriano não foi abolido pelo Concílio Vaticano II, mas, ao contrário, ratificado por ele como canto próprio para a sagrada liturgia.

Apesar disso, quase cinquenta anos após sua publicação, o {\GS} ainda é pouco acessível para a maioria de nossas paróquias. Acreditamos que um empecilho para uma adoção maior deste repertório por parte de nossos grupos de cantores ainda é a barreira linguística, pois muitos de nossos cantores não dominam suficientemente bem a pronúncia e a compreensão do latim, presente nos cantos desse livro.

\subsection{Histórico do projeto}

A fim de tornar esse material ainda mais acessível para as nossas comunidades, um trabalho de adaptação das antífonas do {\GS} para a língua portuguesa foi iniciado por Lincoln Haas Hein no final do ano de 2016. Na ocasião, foram publicados os repertórios para as missas do Tempo do Advento, Tempo do Natal e seis das oito missas para o Tempo Comum, além da Missa pelos Defuntos e diversas festas e solenidades que fazem parte do Próprio dos Santos. Em 2020, retomamos esse trabalho, elaborando as adaptações das antífonas para o Tempo da Quaresma, Semana Santa, Tempo Pascal, missas VII e VIII para o Tempo Comum, as missas faltantes do Próprio dos Santos e os Comuns, além das Missas Rituais, por Diversas Necessidades e Votivas. Uma vez preenchidas essas lacunas, revisitamos os repertórios previamente publicados por Hein, a fim de conformá-los ao padrão de diagramação definido para este projeto.

No final de 2021, este projeto passou por duas grandes mudanças, a saber:
\begin{itemize}
  \item as partituras, antes publicadas de maneira avulsa, agora foram compiladas em fascículos;

  \item as partituras originais em latim foram incluídas junto às adaptações para a língua portuguesa;
\end{itemize}
Foi nessa época que o projeto passou a contar com a valiosa contribuição do Prof.~André Gaby no processo de revisão das adaptações das antífonas para língua portuguesa.

Em meados de 2022, o projeto passou por outra grande mudança: a partir dos subsídios gentilmente fornecidos pelo Prof.~Dr.~Clayton Dias, passamos a incluir, nas partituras das antífonas originais em latim, os neumas adiastemáticos presentes no antifonário de Hartker, quando disponíveis. Ao mesmo tempo, passamos a incluir nas partituras as neografias gregorianas correspondentes aos neumas adiastemáticos, exceto naqueles pontos em que há uma divergência entre a melodia do {\GS} e a representação adiastemática correspondente\footnote{Diversos trabalhos de pesquisa posteriores à publicação da edição típica de 1975 do {\GS} corrigiram as melodias de várias antífonas, para eMediatioMarkem de acordo com a representação adiastemática do antifonário de Hartker. Destre estes, destacamos o trabalho do gregorianista Mons. Alberto Turco.}.

Este projeto pode ser empregado como um fim em si mesmo, mas será muito mais valioso se empregado como um meio para facilitar o aprendizado das melodias originais do {\GS}, contornando a barreira linguística num primeiro momento, mas visando o fim último de propiciar o canto das antífonas e salmos originais em latim.

\subsection{Sobre a organização e construção do projeto}

Este projeto dividiu o conteúdo do {\GS} em seis fascículos, a saber:
\begin{enumerate}[I:]
  \item Tempo do Advento e Tempo do Natal
  \item Tempo da Quaresma, Tríduo Pascal e Tempo paschal
  \item Tempo Comum e Liturgia dos Defuntos
  \item Próprio dos Santos e Comuns
  \item Missas Rituais, por Diversas Necessidades e Votivas
  \item Cantos para o Ordinário da Missa e outros cantos diversos
\end{enumerate}

Para as adaptações das antífonas, procuramos traduzir livremente, escolhendo textos que mais bem se encaixem à melodia original sem comprometer demais a fidelidade da tradução. Também procuramos reproduzir, na melodia adaptada, os mesmos elementos retóricos presentes nos neumas adiastemáticos para a melodia original em latim. Para tanto, lançamos mão, quando necessário, de processos de diérese\footnote{Quebra dos neumas originais em neumas menores para acomodá-los em um número maior de sílabas no texto adaptado.} e sinérese\footnote{Aglutinação dos neumas originais em neumas maiores, para acomodá-los em um número menor de sílabas no texto adaptado.} dos neumas originais, observadas as seguintes premissas:
\begin{itemize}
  \item Neumas de duas ou mais notas que apresentam algum tipo de alargamento não sofrerão diérese na melodia adaptada.
  \item A sinérese de dois ou mais neumas na melodia adaptada não resultará em um neuma inexistente na notação adiastemática (como aqueles em que a penúltima nota é alargada, mas a última nota não o é.)
\end{itemize}

Para favorecer a prosódia da língua portuguesa, empregamos frequentemente as elisões\footnote{Aglutinação de duas ou mais vogais consecutivas de palavras distintas, como se pertencecessem a uma única sílaba.} nos textos fixos, como antífonas e hinos. Por outro lado, não empregamos elisões aos versos salmódicos, delegando-as à livre interpretação do cantor.

Por uma questão de completeza, incluímos as adaptações não só das antífonas de entrada, ofertório e comunhão, mas também dos salmos responsoriais e aleluiáticos, bem como do Aleluia com seus versículos, tratos, hinos e sequências. Para estes últimos, procuramos utilizar os textos oficiais presentes na tradução da 2ª edição típica do Missal Romano, publicada pela CNBB, exceto quanto a diferença de métrica entre o hino original e a sua tradução oficial inviabilizem a acomodação melódica.

Para os versos salmódicos em vernáculo, utilizamos os textos do \emph{Saltério Litúrgico da CNBB}, presentes na Liturgia das Horas e no Lecionário. Nos casos em que o texto em questão não se encontra no Saltério Litúrgico (exemplo: Cântico dos Cânticos), tomamos o texto da \emph{Bíblia Sagrada: Tradução Oficial da CNBB}.

Devido à estrutura peculiar dos versos salmódicos no Saltério Litúrgico, por vezes um único versículo está estruturado em dois ou três versos salmódicos que devem ser cantados em sequência, antes de alternar para a antífona. Neste caso, os versos salmódicos subsequentes, que eMediatioMarkão sinalizados com o símbolo \Sequitur{}, devem ser cantados sem a entonação inicial.

\AllowPageFlush

\subsection{Acomodação melódica para versos salmódicos terminados em palavras oxítonas}

Como praticamente não existem palavras oxítonas na língua latina, não há uma forma única de acomodar as melodias dos tons salmódicos a versos em língua portuguesa terminados em palavras oxítonas. Apresentaremos a seguir as duas formas mais comuns de aplicar a acomodação melódica nestes casos, que denominaremos aqui \emph{acomodação latinizada} e \emph{acomodação sinerética}:

\begin{description}
  \item[Acomodação latinizada:] caso a última palavra da frase tenha apenas duas sílabas, o último acento melódico recai sobre a sua penúltima sílaba; caso tenha três ou mais sílabas, o último acento melódico recai sobre a sua antepenúltima sílaba.

  \item[Acomodação sinerética:] o último acento melódico recai, juntamente com os neumas pós-acento, sobre a última sílaba da última palavra, num processo de sinérese dos neumas.
\end{description}

A seguir, apresentamos exemplos de acomodações melódicas latinizadas (à esquerda) e sineréticas (à direita), para cada um dos oito modos utilizados no {\GS}:

\gresetinitiallines{0}

\subsubsection*{I Modo}

\begin{paracol}{2}
  \gabcsnippet{(c4) <i>Ben</i>(f)<i>di</i>(gh)to(h) se(h)ja(h) o(h) Se(h)nhor(h) <b>Deus</b>(ixir1) de(h) <b>Is</b>(gr1)ra(h)el,(h) *(:) por(h)que(h) a(h) seu(h) po(h)vo(h) vi(h)si(h)<i>tou</i>(g) <i>e</i>(f) <b>li</b><alt>1 a</alt>(gr1)ber(h)tou.(h) (::) vi(h)si(h)<i>tou</i>(g) <i>e</i>(f) <b>li</b><alt>1 a 2</alt>(gr1)ber(g)tou.(gh) (::) vi(h)si(h)<i>tou</i>(g) <i>e</i>(f) <b>li</b><alt>1 g</alt>(gr1h)ber(g)tou.(g) (::) vi(h)si(h)<i>tou</i>(g) <i>e</i>(f) <b>li</b><alt>1 g 2</alt>(gr1)ber(g)tou.(ghg) (::) vi(h)si(h)<i>tou</i>(g) <i>e</i>(f) <b>li</b><alt>1 g 4</alt>(hr1)ber(g)tou.(g) (::) vi(h)si(h)<i>tou</i>(g) <i>e</i>(f) <b>li</b><alt>1 f</alt>(gr1h)ber(g)tou.(gf) (::) vi(h)si(h)<i>tou</i>(g) <i>e</i>(f) <b>li</b><alt>1 D 2</alt>(g[ocba:1{])<b>ber</b>(gf[ocba:0}])tou.(d) (::)}

  \switchcolumn

  \gabcsnippet{(c4) <i>Ben</i>(f)<i>di</i>(gh)to(h) se(h)ja(h) o(h) Se(h)nhor(h) Deus(h) de(h) <b>Is</b>(ixir1)ra(h)<b>el</b>,(gr1h) *(:) por(h)que(h) a(h) seu(h) po(h)vo(h) vi(h)si(h)tou(h) e(h) <i>li</i>(g)<i>ber</i>(f)<b>tou</b>.<alt>1 a</alt>(gr1h) (::) e(h) <i>li</i>(g)<i>ber</i>(f)<b>tou</b>.<alt>1 a 2</alt>(gr1gh) (::) e(h) <i>li</i>(g)<i>ber</i>(f)<b>tou</b>.<alt>1 g</alt>(gr1hg) (::) e(h) <i>li</i>(g)<i>ber</i>(f)<b>tou</b>.<alt>1 g 2</alt>(gr1ghg) (::) e(h) <i>li</i>(g)<i>ber</i>(f)<b>tou</b>.<alt>1 g 4</alt>(hr1g) (::) e(h) <i>li</i>(g)<i>ber</i>(f)<b>tou</b>.<alt>1 f</alt>(gr1hGF) (::) e(h) <i>li</i>(g)<i>ber</i>(f)<b>tou</b>.<alt>1 D 2</alt>(gvFD) (::)}
\end{paracol}

\subsubsection*{II Modo}

\begin{paracol}{2}
  \gabcsnippet{(f3) <i>Ben</i>(e)<i>di</i>(f)to(h) se(h)ja(h) o(h) Se(h)nhor(h) Deus(h) de(h) <b>Is</b>(ir1)ra(h)el,(h) *(:) por(h)que(h) a(h) seu(h) po(h)vo(h) vi(h)si(h)tou(h) <i>e</i>(g) <b>li</b><alt>II D</alt>(er1)ber(f)tou.(f) (::)}

  \switchcolumn

  \gabcsnippet{(c4) <i>Ben</i>(e)<i>di</i>(f)to(h) se(h)ja(h) o(h) Se(h)nhor(h) Deus(h) de(h) Is(h)ra(h)<b>el</b>,(ir1h) *(:) por(h)que(h) a(h) seu(h) po(h)vo(h) vi(h)si(h)tou(h) e(h) li(h)<i>ber</i>(g)<b>tou</b>.<alt>II D</alt>(er1f) (::)}
\end{paracol}

\AllowPageFlush

\subsubsection*{III Modo}

\begin{paracol}{2}
  \gabcsnippet{(c4) <i>Ben</i>(g)<i>di</i>(hj)to(j) se(j)ja(j) o(j) Se(j)nhor(j) <b>Deus</b>(kr1) de(j) <b>Is</b>(j[ocba:1{])<b>ra</b>(ih[ocba:0}])el,(j) *(:) por(j)que(j) a(j) seu(j) po(j)vo(j) vi(j)si(j)tou(j) <i>e</i>(h) <b>li</b><alt>III a</alt>(jr1)ber(j)tou.(ih) (::) vi(j)si(j)<i>tou</i>(ji) <i>e</i>(hi) <b>li</b><alt>III g</alt>(hr1)ber(g)tou.(g) (::)}

  \switchcolumn

  \gabcsnippet{(c4) <i>Ben</i>(g)<i>di</i>(hj)to(j) se(j)ja(j) o(j) Se(j)nhor(j) Deus(j) de(j) <b>Is</b>(kr1)ra(j)<b>el</b>,(ir1hj) *(:) por(j)que(j) a(j) seu(j) po(j)vo(j) vi(j)si(j)tou(j) e(j) li(j)<i>ber</i>(h)<b>tou</b>.<alt>III a</alt>(jr1ih) (::) e(j) <i>li</i>(ji)<i>ber</i>(hi)<b>tou</b><alt>III g</alt>(hr1g) (::)}
\end{paracol}

\subsubsection*{IV Modo}

\begin{paracol}{2}
  \gabcsnippet{(c4) <i>Ben</i>(h)<i>di</i>(gh)to(h) se(h)ja(h) o(h) Se(h)nhor(h) <i>Deus</i>(g) <i>de</i>(h) <b>Is</b>(ir1)ra(h)el,(h) *(:) por(h)que(h) a(h) seu(h) po(h)vo(h) vi(h)<i>si</i>(g)<i>tou</i>(h) <i>e</i>(ih) <b>li</b><alt>IV E</alt>(g[ocba:1{])<b>ber</b>(gf[ocba:0}])tou.(e) (::)}

  \switchcolumn

  \gabcsnippet{(c4) <i>Ben</i>(h)<i>di</i>(gh)to(h) se(h)ja(h) o(h) Se(h)nhor(h) Deus(h) de(h) <i>Is</i>(g)<i>ra</i>(h)<b>el</b>,(ir1h) *(:) por(h)que(h) a(h) seu(h) po(h)vo(h) vi(h)si(h)tou(h) <i>e</i>(g) <i>li</i>(h)<i>ber</i>(ih)<b>tou</b>.<alt>IV E</alt>(gr1FE) (::)}

  \switchcolumn*

  \gabcsnippet{(c3) <i>Ben</i>(i)<i>di</i>(hi)to(i) se(i)ja(i) o(i) Se(i)nhor(i) <i>Deus</i>(h) <i>de</i>(i) <b>Is</b>(jr1)ra(i)el,(i) *(:) por(i)que(i) a(i) seu(i) po(i)vo(i) vi(i)<i>si</i>(h)<i>tou</i>(i) <i>e</i>(j) <b>li</b><alt>IV d</alt>(hr1)ber(i)tou.(i) (::) vi(i)si(i)tou(i) e(i) <b>li</b><alt>IV c</alt>(ir1)ber(h)tou.(h) (::) vi(i)<i>si</i>(h)<i>tou</i>(i) <i>e</i>(j) <b>li</b><alt>IV A</alt>(hr1)ber(f)tou.(f) (::)}

  \switchcolumn

  \gabcsnippet{(c3) <i>Ben</i>(i)<i>di</i>(hi)to(i) se(i)ja(i) o(i) Se(i)nhor(i) Deus(i) de(i) <i>Is</i>(h)<i>ra</i>(i)<b>el</b>,(jr1i) *(:) por(i)que(i) a(i) seu(i) po(i)vo(i) vi(i)si(i)tou(i) <i>e</i>(h) <i>li</i>(i)<i>ber</i>(j)<b>tou</b>.<alt>IV d</alt>(hr1i) (::) vi(i)si(i)tou(i) e(i) li(i)ber(i)<b>tou</b>.<alt>IV c</alt>(ir1h) (::) vi(i)si(i)tou(i) <i>e</i>(h) <i>li</i>(i)<i>ber</i>(j)<b>tou</b>.<alt>IV A</alt>(hr1f) (::)}
\end{paracol}

\subsubsection*{V Modo}

\begin{paracol}{2}
  \gabcsnippet{(c3) <i>Ben</i>(d)<i>di</i>(f)to(h) se(h)ja(h) o(h) Se(h)nhor(h) Deus(h) de(h) <b>Is</b>(ir1)ra(h)el,(h) *(:) por(h)que(h) a(h) seu(h) po(h)vo(h) vi(h)si(h)<b>tou</b><alt>V a</alt>(ir1) e(g) <b>li</b>(hr1)ber(f)tou.(f) (::)}

  \switchcolumn

  \gabcsnippet{(c3) <i>Ben</i>(d)<i>di</i>(f)to(h) se(h)ja(h) o(h) Se(h)nhor(h) Deus(h) de(h) Is(h)ra(h)<b>el</b>,(ir1h) *(:) por(h)que(h) a(h) seu(h) po(h)vo(h) vi(h)si(h)tou(h) e(h) <b>li</b><alt>V a</alt>(ir1)ber(g)<b>tou</b>.(hr1f) (::)}
\end{paracol}

\subsubsection*{VI Modo}

\begin{paracol}{2}
  \gabcsnippet{(c4) <i>Ben</i>(f)<i>di</i>(gh)to(h) se(h)ja(h) o(h) Se(h)nhor(h) Deus(h) <i>de</i>(g) <b>Is</b>(hr1)ra(f)el,(f) *(:) por(h)que(h) a(h) seu(h) po(h)vo(h) vi(h)si(h)<i>tou</i>(f) <i>e</i>(gh) <b>li</b><alt>VI F</alt>(gr1)ber(f)tou.(f) (::)}

  \switchcolumn

  \gabcsnippet{(c4) <i>Ben</i>(f)<i>di</i>(gh)to(h) se(h)ja(h) o(h) Se(h)nhor(h) Deus(h) de(h) Is(h)<i>ra</i>(g)<b>el</b>,(hr1f) *(:) por(h)que(h) a(h) seu(h) po(h)vo(h) vi(h)si(h)tou(h) e(h) <i>li</i>(f)<i>ber</i>(gh)<b>tou</b>.<alt>VI F</alt>(gr1f) (::)}
\end{paracol}

\subsubsection*{VII Modo}

\begin{paracol}{2}
  \gabcsnippet{(c3) <i>Ben</i>(hg)<i>di</i>(hi)to(i) se(i)ja(i) o(i) Se(i)nhor(i) <b>Deus</b>(kr1) de(j) <b>Is</b>(ir1)ra(j)el,(j) *(:) por(i)que(i) a(i) seu(i) po(i)vo(i) vi(i)si(i)<b>tou</b>(jr1) e(i) <b>li</b><alt>VII d</alt>(hr1)ber(h)tou.(gi) (::) vi(i)si(i)<b>tou</b>(jr1) e(i) <b>li</b><alt>VII c</alt>(hr1)ber(h)tou.(gh) (::) vi(i)si(i)<b>tou</b>(jr1) e(i) <b>li</b><alt>VII c 2</alt>(hr1)ber(h)tou.(ih) (::) vi(i)si(i)<b>tou</b>(jr1) e(i) <b>li</b><alt>VII a</alt>(hr1)ber(h)tou.(gf) (::)}

  \switchcolumn

  \gabcsnippet{(c3) <i>Ben</i>(hg)<i>di</i>(hi)to(i) se(i)ja(i) o(i) Se(i)nhor(i) Deus(i) de(i) <b>Is</b>(kr1)ra(j)<b>el</b>,(ir1j) *(:) por(i)que(i) a(i) seu(i) po(i)vo(i) vi(i)si(i)tou(i) e(i) <b>li</b>(jr1)ber(i)<b>tou</b>.<alt>VII d</alt>(hr1gi) (::) e(i) <b>li</b>(jr1)ber(i)<b>tou</b>.<alt>VII c</alt>(hr1gh) (::) e(i) <b>li</b>(jr1)ber(i)<b>tou</b>.<alt>VII c 2</alt>(hr1ih) (::) e(i) <b>li</b>(jr1)ber(i)<b>tou</b>.<alt>VII a</alt>(hr1vGF) (::)}
\end{paracol}

\subsubsection*{VIII Modo}

\begin{paracol}{2}

  \gabcsnippet{(c4) <i>Ben</i>(g)<i>di</i>(h)to(j) se(j)ja(j) o(j) Se(j)nhor(j) Deus(j) de(j) <b>Is</b>(kr1)ra(j)el,(j) *(:) por(j)que(j) a(j) seu(j) po(j)vo(j) vi(j)si(j)<i>tou</i>(h) <i>e</i>(j) <b>li</b><alt>VIII c</alt>(kr1)ber(j)tou.(j) (::) vi(j)si(j)<i>tou</i>(i) <i>e</i>(j) <b>li</b><alt>VIII g</alt>(hr1)ber(g)tou.(g) (::)}

  \switchcolumn

  \gabcsnippet{(c4) <i>Ben</i>(g)<i>di</i>(h)to(j) se(j)ja(j) o(j) Se(j)nhor(j) Deus(j) de(j) Is(j)ra(j)<b>el</b>,(kr1j) *(:) por(j)que(j) a(j) seu(j) po(j)vo(j) vi(j)si(j)tou(j) e(j) <i>li</i>(h)<i>ber</i>(j)<b>tou</b>.<alt>VIII c</alt>(kr1j) (::) e(j) <i>li</i>(i)<i>ber</i>(j)<b>tou</b>.<alt>VIII g</alt>(hr1g) (::)}
\end{paracol}

\gresetinitiallines{1}

Ambas as acomodações são igualmente válidas, mas, por uma questão de escolha editorial, adotaremos por padrão neste fascículo a \textbf{acomodação latinizada}, ficando a critério do leitor a substituição pela acomodação sinerética, conforme a conveniência.

\AllowPageFlush

\subsection{Tabela de correspondência entre a notação gregoriana e a notação utilizada neste projeto}

\begin{center}
  \begin{tabular}{|c|c|c|}
    \hline
    & Notação gregoriana & Notação utilizada neste projeto \\
    \hline
    Episema &
    \raisebox{-2mm}{\begin{minipage}[c]{3cm}
      \gabcsnippet{(c4) (g_) (:) (f_h) (:) (h_f_) (::)}
    \end{minipage}} &
    \begin{minipage}[c]{8cm}
      \lilypondfile[staffsize=24]{snippets/epi.ly}
    \end{minipage} \\
    \hline
    Liquescência &
    \raisebox{-2mm}{\begin{minipage}[c]{3.5cm}
      \gabcsnippet{(c4) (hf~) (:) (fh~) (:) (g>) (:) (g<) (::)}
    \end{minipage}} &
    \begin{minipage}[c]{8cm}
      \lilypondfile[staffsize=24]{snippets/liq.ly}
    \end{minipage} \\
    \hline
    Quilisma &
    \raisebox{-2mm}{\begin{minipage}[c]{2cm}
      \gabcsnippet{(c4) (ewf) (:) (f!gwh) (::)}
    \end{minipage}} &
    \begin{minipage}[c]{5cm}
      \lilypondfile[staffsize=24]{snippets/quil.ly}
    \end{minipage} \\
    \hline
    Oriscus &
    \raisebox{-2mm}{\begin{minipage}[c]{1.3cm}
      \gabcsnippet{(c4) (go) (::)}
    \end{minipage}} &
    \begin{minipage}[c]{2cm}
      \lilypondfile[staffsize=24]{snippets/oris.ly}
    \end{minipage} \\
    \hline
  \end{tabular}
\end{center}

\AllowPageFlush

%\subsection{Lista de acrônimos presentes nas adaptações para o vernáculo}

%\begin{center}
%  \begin{tabular}{rl}
%    HL-CNBB & Hinário Litúrgico (CNBB)                   \\
%    LH-CNBB & Liturgia das Horas (CNBB, 2ª Edição, 2004) \\
%    MR-CNBB & Missal Romano (CNBB, 2ª Edição, 1997)      \\
%  \end{tabular}
%\end{center}

\AllowPageFlush

\tocsection{Instrução Geral}\label{section:praenotanda}

\begin{center}
  \begin{rubrica}
    Texto adaptado de duas traduções para língua portuguesa, gentilmente cedidas pelo Prof.\ Dr.\ Clayton Dias e por Luís Augusto Rodrigues Domingues.
  \end{rubrica}
\end{center}

\subsection{I.\@ Da natureza da presente edição}\label{subsection:praenotanda-1}

\FirstPara{1} Desejando desenvolver o canto sagrado e a ativa participação dos fiéis nas sagradas celebrações cantadas, o Sacrossanto Concílio Ecumênico Vaticano II, na Constituição sobre a Sagrada Liturgia, dispôs que, além de se dever terminar a edição típica das melodias gregorianas, fosse preparada ``uma edição contendo melodias mais simples, para uso das igrejas menores'' (n. 117). Cumprindo com o desejo dos Padres do Concílio, alguns especialistas prepararam esta edição para os cantos estabelecidos no  Ordinário e no Próprio da Missa, destinando-a àquelas igrejas que dificilmente podem executar de modo digno as melodias mais ornadas do {\GR}.

\Para{2} O {\GR} seja mantido em grandíssima honra pela Igreja, pelos admiráveis aspectos de piedade e de arte, dos quais é abundantemente dotado, e seja conservado integralmente o seu valor. Por esta razão é desejável que, segundo a nova estruturação dele mesmo a partir do \emph{Ordo Cantus Missæ} (\emph{Typis Polyglottis Vaticanis} 1972), seja oportunamente utilizado naquelas igrejas que são dotadas de uma {\ScholaC} preparada com a necessária formação técnica para poder executar dignamente as melodias mais ornadas.

Na verdade, recomenda-se que também nas igrejas menores, que usam o {\GS}, sejam  adotadas algumas partes tomadas do {\GR}, particularmente aquelas mais fáceis ou aquelas que são mais tradicionalmente usadas entre as pessoas.

\Para{3} Por este motivo, não é obrigatório que o tesouro destes dois livros seja usado separadamente; na verdade, uma certa fusão das melodias extraídas de uma e da outra fonte pode oferecer uma maior e feliz variedade.

\Para{4} Portanto, se for feita uma escolha sábia, o {\GS} não empobrece o tesouro musical das melodias gregorianas, mas, ao contrário, o enriquece. Em primeiro lugar, no que diz respeito à escolha dos textos, porque ele introduz outros textos no uso litúrgico, que até agora não estão contidos no Missal Romano; em segundo lugar, também no que se refere às melodias, no momento em que no {\GS} foram acolhidas e propostas numerosas outras, extraídas do tesouro autêntico das fontes gregorianas; e, em terceiro lugar, também sob o aspecto pastoral, porque oferece a possibilidade de desenvolver celebrações com canto também nos grupos menores.

\subsection{II.\@ Dos critérios adotados}\label{subsection:praenotanda-2}

\FirstPara{5} Para que a celebração eucarística possa ser realizada em todos os lugares de uma forma mais nobre, isto é, com canto, e nela se possa obter a participação dos fiéis, era absolutamente necessário que fossem encontradas melodias simples.

\Para{6} Mas tais melodias simples não poderiam ser extraídas das melodias mais ornadas, contidas no {\GR}, visto que, de modo algum, era lícito retirar notas ou melismas destas últimas; e muito menos parecia apropriado para tal trabalho criar integralmente melodias neogregorianas para os textos do Missal Romano.

\Para{7} Assim, foram extraídas do tesouro do canto gregoriano melodias autênticas, correspondentes ao objetivo, seja das edições típicas já em circulação, seja das fontes manuscritas, tanto do rito romano como dos outros ritos latinos.

\Para{8} Mas desta nova seleção de melodias nasceu também uma nova série de textos: de fato, rarissimamente foi encontrada uma melodia simples unida às mesmíssimas palavras reportadas no Missal. Por isso, toda vez que não foi possível ter uma concordância, foram escolhidas partes que oferecem palavras semelhantes àquelas do Missal Romano, ou pelo menos próximas ao seu significado. Muitas vezes, todavia, o texto da antífona, não podendo ser proposto como antífona, aparece como versículo do próprio salmo a ser cantado após a antífona.

\Para{9} Consequentemente, visto que estes novos textos foram escolhidos somente com base em motivações musicais, de modo algum será lícito usá-los sem notas musicais.

\subsection{III.\@ Dos cantos do {\KS}}\label{subsection:praenotanda-3}

\FirstPara{10} As partes são dispostas de forma a constituir cinco esquemas simples do Ordinário da Missa, mas que não possuem por si mesmas nenhuma relação com o grau de festividade dos dias litúrgicos. O primeiro destes esquemas responde de modo particular ao desejo do Concílio: ``Tomem-se providências para que os fiéis possam rezar ou cantar juntos, mesmo em latim, as partes do Ordinário que lhes competem.'' (SC n.\ 54). Todavia, as partes individuais são designadas com números progressivos, com o objetivo de facilitar, se desejável, a composição do Ordinário da Missa com partes extraídas dos diversos esquemas.

\Para{11} Ao final do hino \textcolor{gregoriocolor}{\emph{Glória in excélsis}}, retomado do rito ambrosiano, foi acrescentado um \textcolor{gregoriocolor}{\emph{Amen}} simples, mas este não está em desacordo com a genuína tradição mais antiga.

\Para{12} Nada impede que o povo não cante todo o \textcolor{gregoriocolor}{\emph{Agnus Dei}}, mas somente responda com as palavras \textcolor{gregoriocolor}{\emph{miserére nobis}} e \textcolor{gregoriocolor}{\emph{dona nobis pacem}}.

\subsection{IV.\@ Da forma dos cantos da Missa}

\FirstPara{13} Para os cantos de entrada, de ofertório e de comunhão foi utilizada a forma apropriada que consiste em uma antífona a ser repetida após o versículo de um salmo.

\Para{14} Para os cantos que ocorrem entre as leituras, segundos os diversos tempos do ano, encontram-se:
\begin{enumerate}[a)]
  \item o salmo responsorial, com o responsório salmódico ou aleluiático;

  \item o salmo sem responsório, que tradicionalmente é chamado de \emph{tractus};

  \item o {\Al} com alguns versículos do salmo no tempo em que se canta o \textcolor{gregoriocolor}{\Al}, ou uma outra aclamação não aleluiática ao Evangelho no Tempo da Quaresma.
\end{enumerate}

\subsection{V.\@ Da estrutura dos esquemas das Missas}\label{subsection:praenotanda-5}

\FirstPara{15} No Próprio do Tempo, com exceção da Quaresma, não há cantos próprios para cada domingo, mas para cada tempo litúrgico são oferecidos um ou vários esquemas com a faculdade de adotá-los nos domingos de qualquer tempo.

Todavia, cada celebração do Senhor possui cantos próprios.

\Para{16} No Próprio dos Santos, encontram-se os cantos da Missa próprios para as celebrações que tem precedência sobre os domingos.

\Para{17} Os Comuns dos Santos são organizados no mesmo modo dos Comuns do Missal Romano; todavia, normalmente para cada uma das ordens dos santos é oferecido somente um esquema, mas com vários cantos para as diversas partes da Missa de modo que se possa escolher um ou outro que melhor se adapte ao Santo.

\subsection{VI.\@ Das pessoas requeridas para a execução dos cantos do {\GS}}

\FirstPara{18} Tendo presente o princípio proposto na Constituição sobre a Liturgia, que ``nas celebrações litúrgicas, seja quem for, ministro ou fiel, exercendo o seu ofício, faça tudo e só aquilo que pela natureza da coisa ou pelas normas litúrgicas lhe compete'' (SC n.\ 28), pela estrutura do {\GS}:
\begin{enumerate}[a)]
  \item o cantor entoa as antífonas e propõe os versículos dos salmos, ao que o povo responde. O salmo pode ser cantado também pela {\Schola}.

  \item a assembleia de fiéis deve cantar as antífonas e os responsórios dos salmos entre as leituras. A parte dos fiéis às vezes pode ser confiada também à {\Schola}; contudo, convém que ao menos os responsórios dos salmos entre as leituras, dado o seu caráter e a facilidade com a qual podem ser cantados, sejam executados por toda a assembleia.
\end{enumerate}

\subsection{VII.\@ Do uso do {\GS}}
\FirstPara{19} Para a entrada, o ofertório e a comunhão, canta-se a antífona com um ou vários versículos do salmo, conforme a ocasião.

A antífona é repetida depois do versículo do salmo; porém, os versículos podem ser escolhidos livremente dentre aqueles que são propostos, não se omitindo aqueles que são necessários para que a frase tenha um sentido completo. Para a entrada e a comunhão, como conclusão, pode ser cantado o \textcolor{gregoriocolor}{\emph{Gloria Patri}} e o \textcolor{gregoriocolor}{\emph{Sicut erat}}, combinados num único versículo, como é indicado nos tons comuns\footnote{Em nosso projeto, os tons comuns para o \textcolor{gregoriocolor}{\emph{Gloria Patri}} estão disponíveis no fascículo 6.}.

Ao cantar o salmo, sejam respeitados dois elementos particulares do tom salmódico: o início, onde o final da antífona se conecta com a corda de récita do salmo; e a terminação, onde o final do tom salmódico se conecta com o início da antífona.

\Para{20} Os cantos entre as leituras sejam ordenados deste modo:

Quando são realizadas duas leituras antes do Evangelho, os cantos são ordenados assim:
\begin{enumerate}[I)]
  \item Fora do Tempo da Quaresma e do Tempo Pascal, depois da primeira leitura canta-se o salmo responsorial; depois da segunda leitura: ou o salmo aleluiático ou a antífona \textcolor{gregoriocolor}{\Al} com os seus versículos.

  \item Na Quaresma, depois da primeira leitura: o primeiro salmo responsorial; depois da segunda: ou o segundo salmo responsorial, ou a antífona de aclamação ou o \emph{tractus}.

  \item No Tempo Pascal, depois da primeira leitura: o primeiro ou o segundo salmo aleluiático; depois da segunda leitura: ou o segundo salmo aleluiático, ou a antífona \textcolor{gregoriocolor}{\Al} com os seus versículos.
\end{enumerate}

Toda vez que é feita somente uma leitura antes do Evangelho, realiza-se somente um canto à escolha entre os cantos apropriados.

Do salmo, cantam-se ao menos cinco versículos, tomados à escolha, quando são propostos vários.

\Para{21} Quando para o mesmíssimo tempo são oferecidos vários esquemas, pode-se escolher à vontade um ou outro, segundo aquilo que parece mais oportuno. Na verdade, pode-se também tomar partes de um e partes de outros.

Para a comunhão pode-se cantar sempre o Salmo 33 \textcolor{gregoriocolor}{\emph{Benedícam Dóminum}}, com o \Rbar. \textcolor{gregoriocolor}{\Al} ou \textcolor{gregoriocolor}{\emph{Gustáte}}. Na verdade, pode-se também cantar à vontade outros cantos adequados, como vem indicado no final do livro, p. \textcolor{gregoriocolor}{462}\footnote{Em nosso projeto, os cânticos de comunhão à escolha estão presentes no fascículo 6.}.