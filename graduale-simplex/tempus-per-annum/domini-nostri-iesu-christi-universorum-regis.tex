% !TeX root = ../fasciculo-3/a4.tex
% chktex-file 1
\subsection{Antífona de Entrada}\label{subsection:tempus-per-annum/domini-nostri-iesu-christi-universorum-regis/introitus}
\MakeChantAntiphon{regnum-eius.8c}
\nobreaksubsection{Salmo 71(72)}
\MakeChantPsalm{psalmi-ad-introitum}{regnum-eius.8c}

\AllowPageFlush

\subsection[Salmo Responsorial]{Salmo Responsorial}\label{subsection:tempus-nativitatis/in-epiphania-domini/psalmus-responsorius}
\subsubsection{Salmo 71(72)}
\MakeChantPsalmMulti{psalmi-responsorii}{adorabunt-eum.E5}{
  {psalmus-v1}{psalmus-v1-pt},
  {psalmus-v2}{psalmus-v2-pt}
}

\AllowPageFlush

\subsection{Aleluia}\label{subsection:tempus-nativitatis/in-epiphania-domini/alleluia}
\MakeChantAlleluia{3g}
\nobreaksubsection{Salmo 28(29)}
\MakeChantPsalm{psalmi-ad-alleluia}{afferte-domino.3g}

\AllowPageFlush

\subsection[Salmo Aleluiático]{Salmo Aleluiático}\label{subsection:tempus-nativitatis/in-epiphania-domini/psalmus-alleluiaticus}
\subsubsection{Salmo 28(29)}
\begin{rubrica}
  O primeiro {\normalfont\Rbar} pode ser cantado apenas pelo grupo de cantores ou por todos. O segundo {\normalfont\Rbar} é cantado por todos.
\end{rubrica}
\MakeChantPsalm{psalmi-alleluiatici}{afferte-domino.C4}

\AllowPageFlush

\subsection{Antífona de Ofertório}\label{subsection:tempus-per-annum/domini-nostri-iesu-christi-universorum-regis/offertorium}
\MakeChantAntiphon{ecce-dedi-te.8G}
\nobreaksubsection{Salmo 46(47)}
\MakeChantPsalm{psalmi-ad-offertorium}{ecce-dedi-te.8G}

\AllowPageFlush

\subsection{Antífona de Comunhão}\label{subsection:tempus-per-annum/domini-nostri-iesu-christi-universorum-regis/communio}
\MakeChantAntiphon{magnificabitur.7a}
\nobreaksubsection{Cântico de Davi (1Cr 29)}
\MakeChantPsalm{psalmi-ad-communionem}{magnificabitur.7a}