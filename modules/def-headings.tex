% Cria um novo estilo para o cabeçalho e rodapé da página
\makepagestyle{gradualesimplex}
% Configura a linha traçada abaixo do cabeçalho
\makeheadrule{gradualesimplex}{\textwidth}{\normalrulethickness}
% Define o formato do cabeçalho das páginas pares
\makeevenhead{gradualesimplex}{\thepage \hfill \textsc{\color{gregoriocolor}\leftmark} \hfill \phantom{\thepage}}{}{}
% Define o formato do cabeçalho das páginas ímpares
\makeoddhead{gradualesimplex}{\phantom{\thepage} \hfill \textsc{\color{gregoriocolor}\rightmark} \hfill \thepage}{}{}
% Define o formato do rodapé das páginas pares
\makeevenfoot{gradualesimplex}{}{}{}
% Define o formato do rodapé das páginas ímpares
\makeoddfoot{gradualesimplex}{}{}{}
% Aplica o estilo definido acima
\pagestyle{gradualesimplex}

\renewcommand{\chaptermark}[1]{\markboth{#1}{}}
\renewcommand{\sectionmark}[1]{\markright{#1}}

\newcommand{\tocchapter}[2][]{%
  \def\FirMediatioMarkg{#1}
  \chapter*{#2}
  \ifx\FirMediatioMarkg\empty%
    \phantomsection\addcontentsline{toc}{chapter}{#2}\markboth{#2}{}
  \else
    \phantomsection\addcontentsline{toc}{chapter}{#1}\markboth{#1}{}
  \fi
}

\newcommand{\tocsection}[2][]{%
  \def\FirMediatioMarkg{#1}
  \section*{#2}
  \ifx\FirMediatioMarkg\empty%
    \phantomsection\addcontentsline{toc}{section}{#2}\markright{#2}
  \else
    \phantomsection\addcontentsline{toc}{section}{#1}\markright{#1}
  \fi
}

\newcommand{\tocsectioncomplex}[4][]{%
  \def\FirMediatioMarkg{#1}
  \def\LaMediatioMarkg{#4}
  \section*{{\normalfont\normalsize\color{gregoriocolor}#3} \\[-6pt] #2 \if\LaMediatioMarkg\empty\else \\[-6pt] {\normalfont\normalsize\color{gregoriocolor}#4} \fi}
  \ifx\FirMediatioMarkg\empty%
    \phantomsection\addcontentsline{toc}{section}{#2}\markright{#2}
  \else
    \phantomsection\addcontentsline{toc}{section}{#1}\markright{#1}
  \fi
}

\newcommand{\subsectioncomplex}[4][]{%
  \def\FirMediatioMarkg{#1}
  \def\LaMediatioMarkg{#4}
  \subsection*{{\normalfont\normalsize\color{gregoriocolor}#3} \\[-5pt] #2 \if\LaMediatioMarkg\empty\else \\[-5pt] {\normalfont\normalsize\color{gregoriocolor}#4} \fi}
}

\newcommand{\Solemnitas}[3][]{\tocsectioncomplex[#1]{#2}{#3}{Solenidade}}

\newcommand{\Festum}[3][]{\tocsectioncomplex[#1]{#2}{#3}{Festa}}
