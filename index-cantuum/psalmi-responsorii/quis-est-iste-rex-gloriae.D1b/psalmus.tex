\SetVersePairs{
  {\item \VSup{1b}Dómini est terra et plenitúdo \MediatioD{e}[ius,] orbis terrárum et qui hábitant in \TerminatioDI{e}o.~\Responsorium}%
    {\item \VSup{1}Ao Senhor pertence a terra e o que ela en\MediatioD{cer}[ra,] o mundo inteiro com o seres que o po\TerminatioDI{vo}am.~\Responsorium},
  {\item \VSup{2}Quia ipse super mária fundávit \MediatioD{e}[um,] et super flúmina firmávit \TerminatioDI{e}um.~\Responsorium}%
    {\item \VSup{2}Porque ele a tornou firme sobre os \MediatioD{ma}[res] e sobre as águas a mantém inaba\TerminatioDI{lá}vel.~\Responsorium},
  {\item \VSup{3}Quis ascéndet in montem \MediatioD{Dó}[mini,] aut quis stabit in loco sancto \TerminatioDI{e}ius?~\Responsorium}%
    {\item \VSup{3}“Quem subirá até o monte do \MediatioD{Se}[nhor,] quem ficará em sua santa ha\TerminatioDI{bi}tação?''~\Responsorium},
  {\item \VSup{4}Innocens mánibus et mundo \Flexa*{cor}[de,] qui non accépit in vanum nomen \MediatioD{e}[ius,] nec iurávit in \TerminatioDI{do}lum.~\Responsorium}%
    {\item \VSup{4}“Quem tem mãos puras e inocente \Flexa*{co}[ração,] quem não dirige sua mente para o \MediatioD{cri}[me,] nem jura falso para o dano do seu \TerminatioDI{pró}ximo.~\Responsorium},
  {\item \VSup{5}Hic accípiet benedictiónem a \MediatioD{Dó}[mino,] et iustificatiónem a Deo salutari \TerminatioDI{su}o.~\Responsorium}%
    {\item \VSup{5}So bre este desce a bênção do \MediatioD{Se}[nhor] e a recompensa de seu Deus e \TerminatioDI{Sal}vador”.~\Responsorium},
  {\item \VSup{6}Hæc est generátio quæréntium \MediatioD{e}[um,] quæréntium fáciem Dei \TerminatioDI{Ia}cob.~\Responsorium}%
    {\item \VSup{6}É assim a geração dos que o pro\MediatioD{cu}[ram,] e do Deus de Israel buscam a \TerminatioDI{fa}ce.~\Responsorium},
  {\item \VSup{7}Attóllite, portæ, cápita \Flexa*{ve}[stra,] et elevámini, portæ æter\MediatioD{\-ná}[les,] et introíbit rex \TerminatioDI{gló}riæ.~\Responsorium}%
    {\item \VSup{7}“Ó portas, levantai vossos \Flexa*{fron}[tões!] Elevai-vos bem mais alto, antigas \MediatioD{por}[tas,] a fim de que o Rei da glória possa \TerminatioDI{en}trar!”~\Responsorium},
  {\item \VSup{10}Quis est iste rex \MediatioD{gló}[riæ?] Dóminus virtútum ipse  est rex \TerminatioDI{gló}riæ.~\Responsorium}%
    {\item \VSup{10}Dizei-nos: “Quem é este Rei da \Flexa*{gló}[ria?''] “O Rei da glória é o Senhor onipo\MediatioD{ten}[te,] o Rei da glória é o Senhor Deus do  u ni\TerminatioDI{ver}so!”~\Responsorium}
}