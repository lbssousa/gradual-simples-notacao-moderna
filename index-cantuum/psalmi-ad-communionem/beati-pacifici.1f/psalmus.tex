\SetVerseAnnotation{Sl 126(125)}




\SetVersePairs{
  %
  {\item \Inchoatio{Tunc}[ ]{re}plétum est gáudi\MediatioI{o}{ os }{no}[strum,] et língua nostra exsul\TerminatioI{ta}{ti}{ó}ne.~\Antiphona}%
    {\item \Inchoatio{En}{cheu}-se de júbilo \MediatioI{nos}{sa }{bo}[ca] e nossa língua, de \TerminatioI{a}{le}{gri}a.~\Antiphona},
  %
  {\item \Inchoatio{Tunc}[ ]{di}cébant \MediatioI{in}{ter }{gen}[tes:] “Magnificávit Dóminus fáce\TerminatioI{re}[ ]{cum}[ ]{e}is”.~\Antiphona}%
    {\item \Inchoatio{En}{tão} se dizia entre \MediatioI{as}{ na}{ções}[:] “O Senhor fez grandes coi\TerminatioI{sas}[ ]{com}[ ]{e}les!”~\Antiphona},
  %
  {\item \Inchoatio{Ma}{gni}ficávit Dóminus fáce\MediatioI{re}{ no}{bís}[cum;] facti su\TerminatioI{mus}[ ]{læ}{tán}tes.~\Antiphona}%
    {\item \Inchoatio{Sim}[, ]{gran}des coisas o Senhor \MediatioI{fez}{ co}{nos}[co:] ficamos cheios de \TerminatioI{a}{le}{gri}a!~\Antiphona},
  %
  {\item \Inchoatio{Con}{vér}te, Dómine, captivi\MediatioI{tá}{tem }{no}[stram,] sicut torrén\TerminatioI{tes}[ ]{in}[ ]{Aus}tro.~\Antiphona}%
    {\item \Inchoatio{Fa}{zei} voltar, Senhor, os nossos \MediatioI{e}{xi}{la}[dos,] como as torren\TerminatioI{tes}[ ]{no}[ ]{Sul}.~\Antiphona},
  %
  {\item \Inchoatio{Qui}[ ]{sé}mi\MediatioI{nant}{ in }{lá}[crimis,] in exsultati\TerminatioI{ó}{ne}[ ]{me}\-tent.~\Antiphona}%
    {\item \Inchoatio{Os}[ ]{que} semeiam \MediatioI{en}{tre }{lá}[grimas] ceifarão com \TerminatioI{a}{le}{\-gri}\-a.~\Antiphona},
  %
  {\item \Inchoatio{E}{ún}tes \MediatioI{i}{bant et }{fle}[bant,] semen spargén\TerminatioI{dum}[ ]{por}{\-tán}\-tes.~\Antiphona}%
    {\item \Inchoatio{Quan}{do} iam, \MediatioI{i}{am cho}{ran}[do,] levando a semente a \TerminatioI{se}{me}{ar}.~\Antiphona},
  %
  {\item \Inchoatio{Ve}{ni}éntes autem vénient in exsul\MediatioI{ta}{ti}{ó}[ne,] portántes maní\TerminatioI{pu}{los}[ ]{su}os.~\Antiphona}%
    {\item \Inchoatio{Ao}[ ]{vol}tarem, voltam com \MediatioI{a}{le}{gri}[a,] trazendo \TerminatioI{os}[ ]{seus}[ ]{fei}xes.~\Antiphona}
}