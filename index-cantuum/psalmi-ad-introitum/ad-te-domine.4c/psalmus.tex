\begin{flushright}
  \begin{tabular}{rl}
    Harmonização: & Theo Flury, Gennaro M. Becchimanzi
  \end{tabular}
\end{flushright}

\begin{music}
  \parindent0pt
  \instrumentnumber{2} % quantidade de instrumentos
  \setstaffs{1}{2} % quantidade de pautas por instrumento
  \setinterinstrument{1}{50pt}
  \setinterstaff{1}{14}
  \setclef{1}{\bass}
  \nobarnumbers % disable bar numbers
  \indivbarrules
  \sepbarrule{2}

  \setlyrics{verse1}{%
    \VSup{2b}Não tri-____ unfem\ sobre\ mim\ os i-ni-mi-gos!
    \VSup{3a}Não\ se\ envergonha\ quem\ em\ vós\ põe\ a\ espe-ran-ça.%
  }

  \setlyrics{verse2}{%
    \VSup{4}Mos-trai{-}-____ me,\ ó\ Senhor,\ vos-sos ca-mi-nhos
    e\ fazei{-}me\ conhecer\ a\ vossa\ es-tra-da!%
  }

  \setlyrics{verse3}{%
    \VSup{5}Vos-sa_ verdade\ me\ oriente\ e\ me\ con-du-za,
    porque\ sois\ o\ Deus\ da\ minha sal-va-ção;_
    em\ vós\ espero,\ ó\ Senhor,\ todos\ os di-as.%
  }

  \setlyrics{verse4}{%
    \VSup{7cd}De mim____ lembrai{-}vos,\ porque\ sois\ mi-se-ri-cór-dia
    e\ sois\ bondade\ sem\ limites,\ ó\ Se-nhor!_%
  }

  \setlyrics{verse5}{%
    \VSup{8}O Se-____ nhor\ é\ piedade\ e re-ti-dão,_
    e\ reconduz\ ao\ bom\ caminho\ os\ peca-do-res.%
  }

  \setlyrics{verse6}{%
    \VSup{9}E-le____ dirige\ os\ humildes na jus-ti-ça,
    e\ aos\ pobres\ ele\ ensina\ o\ seu\ ca-mi-nho.%
  }

  \assignlyrics{2}{verse1,verse2,verse3,verse4,verse5,verse6}

  \startpiece
  \znotes|&\loffset{1.5}{\verses{\lyric*{\textbf{1.}},\lyric*{\textbf{2.}},\lyric*{\textbf{3.}},\lyric*{\textbf{4.}},\lyric*{\textbf{5.}},\lyric*{\textbf{6.}}}}\en%
  \hardspace{4mm}%
  % Inchoatio
  \NOtes%
  \iextraslurud{4}{H}{5}{L}\InchoatioIVALeft|%
  \iextraslurd{6}{c}\InchoatioIVARight&%
  \InchoatioIVChantA%
  \en%
  \nnnotes|&\InchoatioIVChantB\en%
  % Tenor - Flexa
  \allbarrules\showdotbarrule\condotmultibarrule{1}%
  \bar%
  \notes%
  \tslurud{1}{H}{2}{L}\islurd{1}{H}\TenorIVALeft|%
  \tslur{3}{c}\TenorIVARight&%
  \TenorIVChant%
  \sk\sk\sk\sk\sk\sk\sk\sk\sk\sk\sk%
  \en%
  \NOtes%
  \FlexaIVLeft|%
  \FlexaIVRight&%
  \FlexaIVChant%
  \en%
  \znotes|&\verses{,,\FlexaMark}\en%
  \bar%
  \allbarrules\showbarrule\conmultibarrule{1}%
  % Tenor - Mediatio
  \notes%
  \tslur{1}{H}\textraslurud{4}{H}{5}{L}\TenorIVALeft|%
  \textraslurd{6}{c}\TenorIVARight&%
  \chord{Am}\TenorIVChant%
  \sk\sk\sk\sk\sk\sk\sk\sk\sk\sk\sk%
  \en%
  \NOtes|&\MediatioIVChant\en%
  \znotes|&\MediatioMark\en%
  \bar%
  % Tenor - Terminatio
  \znotes|&\loffset{3}{\verses{\lyric*{\textbf{1.}},\lyric*{\textbf{2.}},\lyric*{\textbf{3.}},\lyric*{\textbf{4.}},\lyric*{\textbf{5.}},\lyric*{\textbf{6.}}}}\en%
  \notes%
  \TenorTerminatioIVALeft|%
  \TenorTerminatioIVARight&%
  \chord{Dm}\TenorIVChant%
  \sk\sk\sk\sk\sk\sk\sk\sk\sk\sk\sk\sk\sk\sk\sk\sk\sk%
  \en%
  \NOtes%
  \TerminatioIVcLeft|%
  \TerminatioIVcRight&%
  \TerminatioIVcChant%
  \en%
  \setdoublebar%
  \endpiece
\end{music}

\begin{music}
  \parindent0pt
  \instrumentnumber{2} % quantidade de instrumentos
  \setstaffs{1}{2} % quantidade de pautas por instrumento
  \setinterinstrument{1}{3pt}
  \setinterstaff{1}{10}
  \setclef{1}{\bass}
  \nobarnumbers % disable bar numbers
  \indivbarrules
  \sepbarrule{2}

  \setlyrics{gloria}{%
    Gló-ria ao Pai\ e\ ao\ Filho\ e\ ao\ Espí-ri-to San-to
    Co-mo_ era\ no\ princípio,\ a-go-ra\lyrlink e sem-pre,
    pelos\ séculos\ dos\ séculos,\ a-mém._%
  }

  \assignlyrics{2}{gloria}

  \startpiece
  \znotes|&\roffset{1.5}{\lyric*{\textbf{Opcional:}}}\en%
  \hardspace{17mm}%
  % Inchoatio
  \NOtes%
  \InchoatioIVALeft|%
  \InchoatioIVARight&%
  \InchoatioIVChantA%
  \en%
  \NOtes|&\InchoatioIVChantBDieresis\en%
  \setemptybar\bar%
  % Tenor - Mediatio
  \notes%
  \tslurud{1}{H}{2}{L}\islurud{1}{H}{2}{L}\TenorIVALeft|%
  \tslur{3}{c}\islurd{3}{c}\TenorIVARight&%
  \TenorIVChant%
  \sk\sk\sk\sk\sk\sk\sk\sk%
  \en%
  \setemptybar\bar%
  \NOtes|&\MediatioIVChant\en%
  \znotes|&\MediatioMark\en%
  \bar%
  % Inchoatio
  \NOtes%
  \tslurud{1}{H}{2}{L}\InchoatioIVALeft|%
  \tslur{3}{c}\InchoatioIVARight&%
  \InchoatioIVChantA%
  \en%
  \neumnotes|&\InchoatioIVChantB\en%
  \setemptybar\bar%
  % Tenor - Mediatio
  \notes%
  \tslurud{1}{H}{2}{L}\TenorIVALeft|%
  \tslur{3}{c}\TenorIVARight&%
  \TenorIVChant\sk\sk\sk\sk\sk\sk%
  \en%
  \setemptybar\bar%
  \NOtes|&\MediatioIVChant\en%
  \znotes|&\MediatioMark\en%
  \bar%
  % Tenor - Terminatio
  \notes%
  \TenorTerminatioIVALeft|%
  \TenorTerminatioIVARight&%
  \chord{Dm}\TenorIVChant%
  \sk\sk\sk\sk\sk\sk\sk\sk\sk%
  \en%
  \setemptybar\bar%
  \neumnotes%
  \TerminatioIVcLeft|%
  \TerminatioIVcRight&%
  \lyrlayout{\bfseries}\CCneum{h}{g}%
  \en%
  \setdoublebar%
  \endpiece
\end{music}