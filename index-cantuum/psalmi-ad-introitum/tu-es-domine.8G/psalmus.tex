\SetVerseAnnotation{Sl 16(15),1b.2--3.5--6.8--9.11~\Antiphona}




\SetVersePairs{
  % 2
  {\item \Inchoatio{Di}{xi} Dómino: “Dóminus meus \MediatioVIII{es}[ tu,] bonum mihi \TerminatioVIII{non}[ ]{est}[ ]{si}ne te”.~\Antiphona}%
    {\item \Inchoatio{Eu}[ ]{dis}se ao Senhor: “Vós sois o meu Se\MediatioVIII{nhor}[,] fora de vós não tenho \TerminatioVIII{bem}[ ]{al}{gum}.~\Antiphona},
  % 3
  {\item \Inchoatio{In}[ ]{san}ctos, qui sunt in terra, ínclitos \MediatioVIII{vi}[ros,] omnis volúntas me\TerminatioVIII{a}[ ]{in}[ ]{e}os.~\Antiphona}%
    {\item \Inchoatio{Aos}[ ]{san}tos, que estão na terra, pessoas hon\MediatioVIII{ra}[das,] vai toda a mi\TerminatioVIII{nha}[ ]{es}{ti}ma.~\Antiphona},
  % 5
  {\item \Inchoatio{Dó}{mi}nus pars hereditátis meæ et cálicis \MediatioVIII{me}[i:] tu es qui détines \TerminatioVIII{sor}{tem}[ ]{me}am.~\Antiphona}%
    {\item \Inchoatio{Se}{nhor}, porção de minha herança e minha \MediatioVIII{ta}[ça,] vós tendes na mão a parte \TerminatioVIII{que}[ ]{me}[ ]{ca}be.~\Antiphona},
  % 6
  {\item \Inchoatio{Fu}{nes} cecidérunt mihi in præ\MediatioVIII{clá}ris; ínsuper et heréditas mea speció\TerminatioVIII{sa}[ ]{est}[ ]{mi}hi.~\Antiphona}%
    {\item \Inchoatio{Pa}{ra} mim as cordas caíram em terrenos apra\MediatioVIII{zí}[veis;] sim, minha herança é preciosa \TerminatioVIII{pa}{ra}[ ]{mim}.~\Antiphona},
  % 8
  {\item \Inchoatio{Pro}{po}nébam Dóminum in conspéctu meo \MediatioVIII{sem}[per;] quóniam a dextris est mihi, non \TerminatioVIII{com}{mo}{vé}bor.~\Antiphona}%
    {\item \Inchoatio{Eu}[ ]{sem}pre tinha o Senhor ante meus \MediatioVIII{o}[lhos;] porque ele está à minha direita, não serei \TerminatioVIII{a}{ba}{la}do.~\Antiphona},
  % 9
  {\item \Inchoatio{Prop}{ter} hoc lætátum est cor \Flexa{me}[um,] et exsultavérunt præcórdia \MediatioVIII{me}[a;] ínsuper et caro mea requi\TerminatioVIII{és}{cet}[ ]{in} spe.~\Antiphona}%
    {\item \Inchoatio{Por}[ ]{is}so, alegrou-se meu cora\Flexa{ção}[,] minhas entranhas exul\MediatioVIII{ta}[ram,] e minha carne repousa na \TerminatioVIII{es}{pe}{\-ran}\-ça.~\Antiphona},
  % 11
  {\item \Inchoatio{No}{tas} mihi fácies vias \Flexa{vi}[tæ,] plenitúdinem lætítiæ cum vultu \MediatioVIII{tu}[o,] delectatiónes in déxtera tua us\TerminatioVIII{que}[ ]{in}[ ]{fi}nem.~\Antiphona}%
    {\item \Inchoatio{Vós}[ ]{me} fareis conhecer os caminhos da \Flexa{vi}[da,] a plenitude da alegria com a vossa \MediatioVIII{fa}[ce,] as delícias à vossa direita, \TerminatioVIII{pa}{ra}[ ]{sem}pre.~\Antiphona}
}